 \section{Introdu��o}

  \begin{frame}{Motiva��o}
     \begin{itemize}
      \item Esta proposta � uma contribui��o para \emph{software} livre em sistemas distribu�dos.
      \item Armazenamento de arquivos � um componente essencial na computa��o de alto desempenho.
      \item Codifica��o por apagamento (\emph{Erasure codes}) introduz redund�ncia e tem sido utilizada em sistemas para alcan�ar confiabilidade e redu��o do custo de armazenamento.
     \end{itemize}
  \end{frame}

  \begin{frame}{Motiva��o}
     \begin{itemize}
      \item Alguns sistemas que utilizam codifica��o por apagamento:
	\begin{itemize}
	   \item \emph{NASA's Deep Space Network} para receber sinais e dados de telemetria (\emph{downlinks}) vindos de ve�culos espaciais (\emph{very distant spacecrafts}) e para enviar telecomandos (\emph{uplinks}) para ve�culos espaciais \cite{Almeida:2007, STO:2010, TDD:2010};
%           \item \emph{Digital Fountain} (\emph{multicasting} multim�dia confi�vel)\cite{Byers:1998}
           \item \emph{Delay and Disruption Tolerant Networks}, redes de sensores e redes~\emph{peer-to-peer} \cite{RTAD:2007, Rodrigues:2005, Houri:2009};
           \item armazenamento de grande volume de dados \cite{Kubiatowicz:2000, Weatherspoon:2002, Fan:2009}.
%           \item sistema de arquivos distribu�do do Hadoop (HDFS)~\cite{HDFS-503:2010}
	\end{itemize}
     \end{itemize}
  \end{frame}

  \begin{frame}{Motiva��o}
     \begin{itemize}
      \item O HDFS, por padr�o, implementa alta disponibilidade dos dados via replica��o simples dos blocos de dados. Esta abordagem acarreta um alto custo de armazenamento para garantir que os dados estar�o sempre dispon�veis.
      \item Esfor�os iniciais nessa linha foram feitos utilizando t�cnicas de \emph{Redundant Array of Independent Drives} (RAID)~\cite{Patterson:1988,HDFS-503:2010} e mais recentemente do algoritmo Reed-Solomon (RS)~\cite{Reed:1960,MR-1969:2010,MR-1970:2010}.
     \end{itemize}
  \end{frame}

  \begin{frame}{Objetivos desta proposta}

  \begin{itemize}

     \item avaliar desempenho das camadas de codifica��o do hadoop, o tamanho do armazenamento;
     \item otimizar, estender o c�digo da camada RAID, da camada RS;
     \item incluir novas codifica��es como a codifica��o Tornado e integrar o c�digo atual com o HDFS.

     \end{itemize}
  \end{frame}
