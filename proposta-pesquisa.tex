\documentclass[12pt,a4paper]{article}
\usepackage[brazil]{babel}
\usepackage[latin1]{inputenc}
\usepackage{latexsym}
\usepackage{amssymb}
\usepackage{graphicx}
\usepackage{subfig}


\bibliographystyle{plain}

\usepackage{setspace, fullpage}

% Formato FAPESP

%% O projeto de pesquisa deve ser apresentado de maneira clara e
%% resumida, ocupando no m�ximo 20 p�ginas datilografadas em espa�o
%% duplo. Para propostas encaminhadas atrav�s do Sistema de Apoio a
%% Gest�o (SAGe), deve anexar documento tipo DOC ou PDF de at� 5Mb.

    %% * Resumo (m�ximo 20 linhas);
    %% * Introdu��o e justificativa, com s�ntese da bibliografia fundamental;
    %% * Objetivos;
    %% * Plano de trabalho e cronograma de sua execu��o;
    %% * Material e m�todos;
    %% * Forma de an�lise dos resultados.

%\sloppy

\title{Proposta de Disserta��o de Mestrado\\[1cm] Estudo e Implementa��o de Mecanismos de Codifica��o por Apagamento no Hadoop \emph{File System}}

\author{Celina d'�vila Samogin\\
  Instituto de Computa��o --- UNICAMP}

\hyphenation{da-ni-fi-ca-das o-ri-gi-nal co-di-fi-ca fa-lha-rem fa-lha ou-tros a-tu-a-li-zam}

\begin{document}

\maketitle

\begin{center}
 Orientadora: Profa. Dra. Islene Calciolari Garcia
\end{center}

\thispagestyle{empty}

\doublespace

\begin{abstract}
  Os dados em um sistema distribu�do confi�vel devem estar dispon�veis
  quando for necess�rio. A codifica��o por apagamento (\emph{erasure
    codes}) tem sido utilizada por sistemas para alcan�ar requisitos
  de confiabilidade e de redu��o do custo de armazenamento de dados. O
  Hadoop � um \emph{framework} para execu��o de aplica��es em
  armazenamento distribu�do de grande volume de dados e que pode ser
  constru�do com \emph{commodity hardware}, que � facilmente acess�vel
  e dispon�vel. Esta proposta apresentar� uma an�lise da viabilidade
  da implementa��o pr�tica de t�cnicas de codifica��o por apagamento
  no Hadoop \emph{Distributed File System} (HDFS), as altera��es no
  Hadoop e a efic�cia dessas altera��es. Esta proposta � uma
  contribui��o para \emph{software} livre em sistemas distribu�dos.
\end{abstract}

%% \begin{abstract}
%% The data in a reliable distributed system should be available when needed. Erasure codes have been used by systems to meet reliability requirements and reduce the cost of data storage. The Hadoop is a framework for running applications on distributed storage of large volumes of data and it can be built with commodity hardware, which is easily accessible and available. This proposal will examine the feasibility of practical implementation of erasure coding techniques in Hadoop File System (HFS), changes in Hadoop and effectiveness of those changes. This proposal is a contribution to free software in distributed systems.
%% \end{abstract}




\pagebreak

\section{Introdu��o}

A codifica��o por apagamento (\emph{erasures codes}) introduz
redund�ncia em um sistema de transimiss�o ou armazenamento de dados de
maneira a permitir a detec��o e corre��o de erros. A codifica��o por
apagamento �, desde os anos 70, utilizada pela \emph{NASA's Deep Space
  Network} para receber sinais e dados de telemetria
(\emph{downlinks}) vindos de ve�culos espaciais (\emph{very distant
  spacecrafts}) e para enviar telecomandos (\emph{uplinks}) para
ve�culos espaciais \cite{Almeida:2007, STO:2010, TDD:2010}.

A t�cnica de codifica��o por apagamento pode ser combinada com a
distribui��o de dados entre v�rios dispositivos de armazenamento, o
que permite o aumento da largura de banda e a corre��o de
erros~\cite{Woitaszek:2007, Plank:1997}. Requisitos de confiabilidade
e de redu��o do tamanho do armazenamento podem ser observados em
sistemas que tratam de: \emph{digital fountain} (\emph{multicasting}
multim�dia confi�vel)\cite{Byers:1998}; \emph{Delay and Disruption
  Tolerant Networks}, redes de sensores e redes~\emph{peer-to-peer}
\cite{Rodrigues:2005, RTAD:2007, Houri:2009} e armazenamento de grande
volume de dados \cite{Kubiatowicz:2000, Weatherspoon:2002, Fan:2009},
como tamb�m o sistemas de arquivos distribu�do do Hadoop
(HDFS)~\cite{HDFS-503:2010}.

O HDFS, por padr�o, implementa alta disponibilidade dos dados via
replica��o simples dos blocos de dados. Esta abordagem acarreta um
alto custo de armazenamento para garantir que os dados estar�o sempre
dispon�veis. O objetivo do uso da codifica��o por apagamento no HDFS �
permitir que o espa�o de armazenamento possa ser reduzido sem
prejudicar a disponibilidade dos dados. Esfor�os iniciais nessa linha
foram feitos utilizando t�cnicas de RAID~\cite{HDFS-503:2010} e mais
recentemente do algoritmo Reed-Solomon~\cite{MR-1969:2010}.

Este trabalho pretende avan�ar esta linha de pesquisa a partir dos
seguintes passos:

\begin{itemize}
\item avalia��o de desempenho, ganhos, e custos de diferentes
  estrat�gias de codifica��o por apagamento;

\item implementa��o de otimiza��es ou extens�es para o c�digo que
  atualmente implementa Reed-Solomon, tentando melhorar,
  principalmente, a parte de distribui��o de blocos;

\item implementa��o de novos algoritmos (e.g., Tornado codes) e
  exten��o da interface atual para aceit�-los;

\item integra��o do c�digo atual com o HDFS.

\end{itemize}

O texto a seguir est� organizado da seguinte maneira: a Se��o 2
introduz os conceitos b�sicos da codifica��o por apagamento, a Se��o 3
comenta o \emph{framework} Hadoop e seu sistema de arquivos, a Se��o 4
apresenta os objetivos deste trabalho e a se��o 5 cita as atividades
propostas e o cronograma de execu��o.




\section{Codifica��o por Apagamento}

\begin{frame}{Codifica��o por Apagamento}
     \begin{itemize}
%        \item A codifica��o de mensagens no emissor antes da transmiss�o e a decodifica��o das mensagens (possivelmente danificadas) que chegam ao receptor, possibilita reparar os efeitos de um canal f�sico com ru�dos \cite{Shannon:1948} sem sobrecarregar a taxa de transmiss�o de informa��o ou o \emph{overhead} de armazenamento \cite{Lin:1983}.
        \item Shannon demonstrou essa teoria em artigo do Bell System Technical Journal \cite{Shannon:1948} de 1948.

        \item Existem dois m�todos b�sicos para tratar erros em comunica��o e ambos envolvem a codifica��o de mensagens. A diferen�a est� em como esses c�digos s�o utilizados:
           \begin{itemize}
              \item Em um \emph{repeat request system}, os c�digos s�o utilizados para detectar erros e se estes existirem, � feito um pedido de retransmiss�o.
              \item Com \emph{forward error correction}, os c�digos s�o usados para detectar e corrigir erros.
           \end{itemize}
\end{itemize}
  \end{frame}

\begin{frame}{Codifica��o por Apagamento}
%     \begin{itemize}
%        \item Na Figura~\ref{fig3:fec}, vemos um sistema que utiliza c�digos de bloco.
%\end{itemize}
\begin{figure}[htb]
  \setlength{\unitlength}{0.9cm}
  \begin{center}
  {\begin{picture}(12.5,6)(0,-3)
    \put(0,2){\framebox(3,1){Fonte}}
    \put(3,2.5){\vector(1,0){2}}
    \put(4,2.6){u}
    \put(5,2){\framebox(4,1){{Codificador (n,k)}}}
    \put(9,2.5){\line(1,0){2}}
    \put(10,2.6){v}
    \put(11,2.5){\vector(0,-1){2}}
    \put(9.5,-0.5){\framebox(3,1){Canal}}
    \put(6.8,-0.1){ru�dos}
    \put(8,-0){\vector(1,0){1.5}}
    \put(11,-0.5){\line(0,-1){2}}
    \put(11,-2.5){\vector(-1,0){2}}
    \put(10,-2.4){r}
    \put(0,-3){\framebox(3,1){Destino}}
    \put(5,-2.5){\vector(-1,0){2}}
    \put(4,-2.4){$\mathaccent 94 u$}
    \put(5,-3){\framebox(4,1){{Decodificador (n,k)}}}
   \end{picture}}
  \end{center}
  \caption{C�digos de bloco}
  \label{fig3:fec}
\end{figure}

  \end{frame}

\begin{frame}{Sistema de arquivos distribu�dos armazenando um arquivo - Replica��o Simples}
   \begin{figure}[h]
     \centering
     \includegraphics[scale=.4]{replicacao-pura.jpg}
     \caption{Sistema com replica��o simples \cite{Plank:2004}}
     \label{fig4:srp}
   \end{figure}
\end{frame}

\begin{frame}{Sistema de arquivos distribu�dos armazenando um arquivo - Codifica��o por Apagamento}
   \begin{figure}[h]
     \centering
     \includegraphics[scale=.4]{codigos-RS.jpg}
     \caption{Sistema com codifica��o por apagamento \cite{Plank:2004}}
     \label{fig5:crs}
   \end{figure}
\end{frame}

 \begin{frame}{RAID-5}
    \begin{figure}[hb]
      \centering
      \includegraphics[scale=0.35]{raid5.jpg}
      \caption{RAID-5: 1 bloco de paridade e \emph{stripe} = 4 blocos \cite{MR-2036:2010}}
      \label{fig5:raid5}
    \end{figure}
 \end{frame}

 \begin{frame}{RAID-6}
    \begin{figure}[hb]
      \centering
      \includegraphics[scale=0.32]{raid6.jpg}
      \caption{RAID-6: 2 blocos de paridade e \emph{stripe} = 4 blocos \cite{MR-2036:2010}}
      \label{fig5:raid6}
    \end{figure}
 \end{frame}

\begin{frame}{Codifica��o RS e Tornado}

\begin{figure}[htb]
  \setlength{\unitlength}{0.9cm}
  \begin{center}
  {\begin{picture}(12.5,6)(0,-3)
    \put(6,2){\vector(0,-1){1}}
    \put(6.2,1.2){m}
    \put(2.9,0){\framebox(7,1){{C\'{a}lculo da palavra codificada mG}}}
    \put(6,0){\vector(0,-1){2}}
    \put(6.2,-1.8){c}
    \put(2,-3){\framebox(9,1){{Escreve palavra c (mensagem e redund�ncia)}}}
   \end{picture}}
  \end{center}
  \caption{Algoritmo de codifica��o}
  \label{fig3:feccode}
\end{figure}
\end{frame}

\begin{frame}{Decodifica��o RS e Tornado}

    \begin{figure}[hb]
      \centering
      \includegraphics[scale=0.34]{processo-decodificacao-fec.jpg}
      \caption{Algoritmo de decodifica��o \cite{AF:2010}}
      \label{fig5:fecdeco}
    \end{figure}

\end{frame}

\begin{frame}
 
  {\tiny
\begin{table}
%\singlespacing
  \setlength{\unitlength}{0.9cm}
  \begin{center}
     \begin{tabular}{|p{1.5cm}||p{1cm}||p{1.5cm}||p{2cm}||p{2cm}|}
      \hline

Trabalho & Modelo & Objetivos & Codifica��o & M�tricas \\ \hline
Byers,1998 \cite{Byers:1998} & \emph{Digital Fountain} & Confiabilidade, Efici�ncia & Replica��o, c�digos Tornado e RS & tempo de codifica��o e decodifica��o de um bloco, \emph{bandwidth}, perda de pacotes\\ \hline

Weatherspoon, 2002 \cite{Weatherspoon:2002} & Arquivador central, n�s & Disponibilidade & Replica��o, Codifica��o por Apagamento & tempo m�dio entre falhas, \emph{overhead} de armazenamento, tempo de verifica��o do bloco \\ \hline

Dabek, 2004 \cite{Dabek:2004} & DHT & Efici�ncia & Replica��o, Codifica��o por Apagamento & lat�ncia \\ \hline

Camargo, 2009 \cite{Camargo:2009} & OppStore & Confiabilidade, Disponibilidade & Replica��o  & \emph{bandwidth}, n�mero de mensagens trocadas \\ \hline

Fan, 2009 \cite{Fan:2009}& HDFS & Confiabilidade, Disponibilidade & Replica��o, RAID & \emph{overhead} de armazenamento\\ \hline

Houri, 2009 \cite{Houri:2009} & \emph{Peer-to-peer} & Disponibilidade & Replica��o, Codifica��o por Apagamento & \emph{bandwidth} \\ \hline

Plank, 2009 \cite{Plank:2009} & $k$ discos de dados e $m$ discos de paridade & Efici�ncia & c�digos Tornado, RS e RAID & tempo de codifica��o de um grande arquivo de v�deo e tempo de decodifica��o de um \emph{drive} de dados\\ \hline
Esta proposta & HDFS e MapReduce & Redu��o do custo de armazenamento & Replica��o, c�digos RAID, RS e Tornado & \emph{overhead} de armazenamento, tempo de lat�ncia de leitura de arquivos\\ \hline
    \end{tabular}
\caption{Compara��o entre sistemas de codifica��o por apagamento}
\label{tab1:comp}
  \end{center}
\end{table}
}


\end{frame}


\section{Hadoop}

Atualmente, o Google � uma empresa de consulta e publicidade e � capaz de fornecer os seus servi�os devido a investimentos em armazenamento distribu�do em larga escala e a capacidade de processamento, estes desenvolvidos \emph{in-house}.

Essa capacidade � fornecida por um grande n�mero de PCs, pelo Google File System (GFS), um sistema de arquivos redundantes em \emph{cluster}, pelo sistema operacional GNU/Linux e pelo MapReduce, um \emph{middleware} de processamento paralelo de dados.

Em 2004, um artigo~\cite{Dean:2004}, que foi publicado por
profissionais da Google, prop�s o MapReduce. Em 2006, estes
profissionais, juntamente com Doug Cutting do Yahoo!, formaram um
sub-projeto do Apache Lucene\footnote{http://www.apache.org} que foi
chamado Hadoop\footnote{http://hadoop.apache.org/}.

Mais recentemente, o projeto Apache Hadoop tem desenvolvido uma
reimplementa��o de partes do GFS e MapReduce e muitos grupos da
comunidade de software livre posteriormente abra�aram essa tecnologia,
permitindo-lhes fazer coisas que eles n�o poderiam fazer em m�quinas
individuais. O Hadoop est� dispon�vel em c�digo fonte sob
licenciamento Apache \emph{license} (compat�vel com GPL).

O Hadoop � um \emph{framework} para executar aplica��es em
armazenamento distribu�do de grande volume de dados que pode ser
constru�do com \emph{commodity hardware}, que � facilmente acess�vel e
dispon�vel.  O Hadoop n�o � um \emph{framework} can�nico. Ele foi
projetado para aplica��es que atualizam dados da seguinte forma: uma
escrita e muitas leituras, atrav�s de acessos por \emph{batch}, com
tamanho da ordem de petabytes, organizados de forma n�o estruturada,
com esquema din�mico e integridade baixa.  Uma lista de aplica��es e
organiza��es que usam o Hadoop pode ser encontrada em
\cite{HadoopWiki:2010}.

Em poucas palavras, o Hadoop disponibiliza um armazenamento
compartilhado (HDFS) e um sistema de an�lise (MapReduce) que comp�em o
seu \emph{kernel}.

\subsection{MapReduce}

O MapReduce utiliza algoritmos de ordena��o para reconstruir sua base de dados.  Um bom uso para o MapReduce s�o aplica��es cujos dados s�o escritos uma vez e lidos muitas vezes. S�o dados n�o estruturados como texto ou imagens. O MapReduce tenta colocar esses dados no n� onde s�o feitas as computa��es, desta forma, o acesso aos dados � r�pido, pois � local \cite{White:2009}.

O MapReduce pode resolver problemas gen�ricos, cujos dados podem ser divididos em matrizes de dados, para cada matriz a mesma computa��o necess�ria (sub-problema) e n�o existe necessidade de comunica��o entre as tarefas (sub-problemas). A execu��o de um t�pico \emph{job} do MapReduce pode ser assim descrita:

\begin{itemize}
    \item Itera��o sobre um n�mero grande de registros
    \item Map extrai algo de cada registro (chave, valor)
    \item Rearranjo (\emph{shuffle}) e ordena��o de resultados intermedi�rios por (chave, valor)
    \item Reduce agrega os resultados intermedi�rios
    \item Gera��o da sa�da
\end{itemize}

Um programas para execu��o no HFS/MapReduce que podem ser escritos em v�rias linguagens como Java, Ruby, Python e C++.


\subsection{Arquitetura do Hadoop \emph{Distributed File System}}

Um \emph{cluster} do HDFS � composto por um �nico NameNode, um
servidor-mestre que gerencia o sistema de arquivos e controla o acesso
aos arquivos de clientes. H� uma s�rie de DataNodes, geralmente um por
n� do \emph{cluster}, que gerenciam o armazenamento anexado ao n� em
que s�o executados. A Figura~\ref{fig6:hfs} mostra o NameNode e os
DataNodes.

Uma t�pica arquitetura de rede em dois n�veis para um \emph{cluster}
Hadoop � constru�da por v�rios \emph{racks} interligados por um
comutador como mostra a Figura~\ref{fig5:hc}. Cada \emph{rack} por sua
vez � formado por v�rios n�s (m�quinas) e seus discos, estes tamb�m
interligados por um comutador.

    \begin{figure}[h]
      \centering
      \includegraphics[scale=0.6]{hadoop-cluster.jpg}
      \caption{Arquitetura de rede em dois n�veis para um cluster Hadoop~\cite{Hadoop:2010}}
      \label{fig5:hc}
    \end{figure} 

O NameNode executa opera��es no sistema de arquivos, como \emph{open}, \emph{close}, \emph{rename} de arquivos e de diret�rios.

HDFS disponibiliza espa�o para sistema de arquivos e permite que os
dados do usu�rio sejam armazenados em arquivos. Internamente, um
arquivo � dividido em um ou mais blocos e esses blocos s�o armazenados
em um conjunto de DataNodes. A Figura~\ref{fig7:hfs} mostra DataNodes
e seus blocos. O tamanho \emph{default} de cada bloco � 64MB.

    \begin{figure}[htbp]
      \centering
      \includegraphics[scale=.6]{HDFS-arquitetura-2.jpg}
      \caption{Arquitetura do HDFS \cite{TR-IC-10-24}}
      \label{fig6:hfs}
    \end{figure} 

Os DataNodes respondem aos pedidos de leitura e escrita de clientes do
sistema de arquivos e tamb�m executam a cria��o, elimina��o e
replica��o de blocos sob instru��o do NameNode. O n�mero de r�plicas �
geralmente 3. A 1$^a$ r�plica fica local, no mesmo n� do c�digo do
cliente. A 2$^a$ r�plica fica em um n� em outro \emph{rack} e a 3$^a$
r�plica fica nesse �ltimo \emph{rack} em outro n�. As 2$^a$ e 3$^a$
r�plicas n�o s�o locais ao bloco replicado.

O NameNode e DataNode s�o partes do \emph{software} projetado para
rodar em \emph{commodity hardware}. Essas m�quinas normalmente
executam um sistema operacional GNU/Linux.

HDFS � constru�do usando a linguagem Java. Qualquer m�quina que suporte
Java pode executar o NameNode ou o DataNode \cite{Hadoop:2010}.

\begin{figure}[h]
  \centering
  \includegraphics[scale=.5]{HDFS-arquitetura-replicacao-2.jpg}
  \caption{Arquitetura do HFS - Datanodes e Blocos \cite{White:2009}}
  \label{fig7:hfs}
\end{figure} 

Os protocolos do HDFS usam o protocolo TCP/IP. O cliente fala o
protocolo ClientProtocol com o NameNode atrav�s de uma porta. Os
DataNodes falam o protocolo DataNodeProtocol com o NameNode. Esses
protocolos executam uma \emph{Remote Procedure Call} (RPC). O NameNode
n�o inicia chamadas RPCs. Ele responde a chamadas RPCs feitas pelo
DataNodes e pelos clientes.


\subsection{Codifica��o por apagamento}

Existe uma nova caracter�stica proposta em 2009 para implementa��o de
uma camada de codifica��o por apagamento no Hadoop utilizando
RAID~\cite{HDFS-503:2010} e uma mais recente utilizando c�digos
RS~\cite{MR-1969:2010}.

A vers�o atual do Hadoop utiliza apenas a t�cnica de replica��o
\cite{White:2009} para obter disponibilidade e confiabilidade de
dados. A inclus�o da codifica��o por apagamento ser� feita com o
objetivo de reduzir o tamanho do armazenamento do HDFS.


\section{Proposta}

\subsection{Objetivos}

Esta proposta apresentar� uma an�lise da viabilidade da implementa��o
pr�tica de v�rias t�cnicas de codifica��o por apagamento no HDFS, as
altera��es no Hadoop e a efic�cia dessas altera��es. Esta proposta �
uma contribui��o para \emph{software} livre em sistemas distribu�dos.

O objetivo do uso da codifica��o � reduzir o tamanho do armazenamento
utilizando para isso a redu��o do fator de replica��o de um bloco e
codifica��o de um conjunto de blocos (a partir de um bloco inicial) de
tal modo que a probabilidade de falha de um bloco permane�a a mesma
que antes ou diminua.

Os testes ser�o feitos para mostrar a efic�cia das altera��es no HDFS:
quanto de espa�o em disco foi economizado e o tempo de lat�ncia de
leitura de arquivos. Tamb�m est� prevista a implementa��o e valida��o
em software dos algoritmos RS e Tornado, que possibilitar� a valida��o
da funcionalidade desses algoritmos.

\subsection{M�todos}

Este trabalho ir� estender e alterar c�digo fonte distribu�do sob a
licen�a Apache. Espera-se a intera��o e colabora��o com os
desenvolvedores. Atualmente, Rodrigo Malta Schimdt, ex-aluno do
Instituto de Computa��o da Unicamp e um dos respons�veis pela inclus�o
de t�cnicas de codifica��o por apagamento no HDFS tem contribu�do com
v�rias ideias para a realiza��o deste trabalho.

Os testes poder�o utilizar:

\begin{itemize}
\item m�quinas do Instituto de Computa��o da Unicamp, principalmente
  do LSD (Laborat�rio de Sistemas Distribu�dos);
\item m�quinas do ambiente computacional do CENAPAD-SP (Centro
  Nacional de Processamento de Alto Desempenho em S�o Paulo)
\item a nuvem do AWS (Amazon \emph{Web Services}).
\end{itemize}

Na tabela~\ref{tab2:ccr} o Modelo 1 � modelo atual do HDFS, o Modelo 2
� modelo em implementa��o com caracter�stica HDFS-503
\cite{HDFS-503:2010} e as Propostas 3, 4 e 5 s�o propostas de
implementa��o de algoritmos de codifica��o para este trabalho. A
disponibilidade, a probabilidade de corrup��o de um arquivo e o espa�o
de armazenamento foram adaptadas de \cite{MR-1969:2010} e de
\cite{Bhagwan:2003}. A possibilidade de falhas nas m�quinas est�o
atreladas a uma boa escolha da distribui��o dos blocos pelos
dispositivos de armazenamento.

\begin{table}

{\small

\singlespacing

  \begin{center}
    \begin{tabular}{|p{2.0cm}||p{3cm}||p{3cm}||p{2.5cm}||p{2.0cm}|} \hline

Proposta/ Modelo & Codifica��o\phantom{\Large X} & Disponibilidade & Probabilidade de corrup��o de um arquivo & Espa�o de armazenamento \\ \hline

1\phantom{\Large X}  & sem codifica��o, fator replica��o = $n$ & baixa em rela��o ao espa�o de armazenamento & $O(p^n)$ & $nx$ \\[2pt] \hline

2\phantom{\Large X}  & C�digos RAID, 1 bloco de paridade, stripe = 10 blocos, fator replica��o = 2 & permite falha em 1 m�quina  & $O(p^4)$ & $2.2x$ \\ \hline

3\phantom{\Large X} & C�digos RS RAID, 4 blocos de paridade, stripe = 10 blocos, fator replica��o = 1 & permite falha em 3 m�quinas & $O(p^5)$ & $1.4x$ \\ \hline

4\phantom{\Large X} & C�digos RS, 4 blocos de paridade, fator replica��o = 1, com $n$ m�quinas & permite falha em at� $(3m + 1)$ m�quinas & $\Omega(p^{3m+2})$ & at� $5x$ \\ \hline

5\phantom{\Large X} & C�digos Tornado, 4 blocos de paridade, fator replica��o = 1, com $n$ m�quinas & permite falha em at� $(3m + 1)$ m�quinas & $\Omega(p^{3m+2})$ & at� $5x$ \\ \hline

    \end{tabular}
\caption{Compara��o entre algoritmos de codifica��o por apagamento e replica��o}
\label{tab2:ccr}
  \end{center}

em que:\\
%$O$ e $\Omega$: nota��o assint�tica, significando \emph{upper bound} e \emph{lower bound} respectivamente\\
$p$ = probabilidade de perda do bloco, $0 < p < 1$\\
$x$ = tamanho do armazenamento em disco de um bloco\\
$m$ = n�mero de fragmentos do bloco inicial antes da codifica��o\\
$n = 4m$ � n�mero de blocos codificados a partir a um bloco inicial; o bloco codificado $b_{i}$ est� armazenado na m�quina $d_{i}$, para $1 \leq i \leq m$; o bloco inicial e a primeira r�plica est�o na m�quina $d1$\\
}

\end{table}

\subsection{Forma de An�lise dos Resultados}

N�s poderemos utilizar alguns \emph{ebooks} do Projeto Gutenberg e do
Portal Dom�nio P�blico como entrada de dados de alguns dos testes:

\begin{itemize}
   
\item {Teste de funcionalidade dos algoritmos de codifica��o e de
    decodifica��o RS e de outra codifica��o como Tornado}

     % com dois PCs interconectados (via porta serial, padr�o RS-232 ou
     % via porta USB), sob sistema operacional GNU/Linux e e representar
     % o que acontece de algum modo: arquivo de log, anima��o.

   \item {Teste \emph{cluster} Hadoop  0.21.0} que utiliza apenas replica��o

   \item {Teste \emph{cluster} Hadoop 0.21.0 com caracter�stica
       HDFS-503} que utiliza replica��o e codifica��o por apagamento
     \cite{HDFS-503:2010}

\end{itemize}

% Este \emph{patch} pretende implementar uma camada opcional de
% codifica��o por apagamento sob o HDFS. O objetivo deste \emph{patch} �
% reduzir o volume de armazenamento do HDFS.

% � poss�vel que sejam usados entrada de dados como imagens, desde que
% possam ser disponibilizadas nas m�quinas utilizadas nos testes: dados
% dos laborat�rios do IC-Unicamp e da Embrapa.

Estamos prevendo duas fases de teste:

\begin{description}

   \item [testes de funcionalidade e de inje��o de falhas] testar os algoritmos que criam  os blocos codificados (dados e paridade) e os mant�m; testar os algoritmos que atendem os pedidos de leitura (r�plica); testar os algoritmos que percebem r�plicas indispon�veis e as reconstroem a partir dos blocos codificados (para isso utilizar possivelmente o Zookeeper, um servi�o de coordena��o de processos de aplica��es em sistemas distribu�dos); testar os algoritmos que percebem blocos indispon�veis e reconstroem as r�plicas (se indispon�veis); esta fase ser� executada em ambiente virtualizado;

   \item [testes de desempenho, de tamanho do armazenamento e de inje��o de falhas] obter uma aproxima��o do tamanho do armazenamento (dados e paridade) para conjuntos de arquivos que ocupem espa�o original do tamanho de alguns gigabytes e terabytes; esta fase ser� executada em ambiente o mais real poss�vel.

\end{description}

Os algoritmos de codifica��o e de decodifica��o poder�o permitir parametrizar:

\begin{itemize}

   \item n�mero de peda�os que o bloco original � dividido antes da gera��o dos blocos codificados
   \item n�mero de blocos de paridade (redund�ncia)
   \item fator de replica��o
%\footnote{Este par�metro � atualmente configurado no HFS como dfs.replication.}
\end{itemize}

{\tiny
\begin{enumerate}
\item Cr�ditos do mestrado
\item Exame de qualifica��o do mestrado
\item Revis�o bibliogr�fica
\item Implementa��o
\item Realiza��o de testes
\item Escrita da disserta��o de mestrado
\item Prepara��o de artigo para congresso ou revista
\item Defesa da disserta��o
\end{enumerate}

\begin{center}
\begin{tabular}{|l||c|c|c|c|c||c|c|c|c|c|c|c|}
\hline
&\multicolumn{5}{l||}{2010}&\multicolumn{6}{l|}{2011}&\multicolumn{1}{l|}{2012}\\
\cline{2-13}
Atividade & 3--4 & 5--6 & 7--8 & 9--10 & 11--12 & 1--2 & 3--4 & 5--6 & 7--8 & 9--10 & 11--12 
& 1--2 \\ \hline
\hline\hline
1 & o&  o&  o&  o&  o&   &   &   &   &   &   &  \\ \hline
2 &  &   &   &  o&   &   &   &   &   &   &   &  \\ \hline
3 & o& oo& oo& oo& oo& oo& oo&  o&  o&  o&  o&  \\ \hline
4 &  &   &  o& oo& oo& oo& oo& oo& oo& oo& oo&  \\ \hline
5 &  &  o&  o& oo& oo& oo& oo& oo& oo& oo& oo&  \\ \hline
6 &  & oo& oo& oo& oo& oo&  o&  o&  o&  o&  o&  \\ \hline
7 &  &   &   &   &   &   &   &  o&  o&  o&   &  \\ \hline
8 &  &   &   &   &   &   &   &   &   &   &   &  o\\ \hline
\hline
\end{tabular}
\end{center}
}


% \subsection{Vantagens e Limita��es deste trabalho}

% O modelo estudado neste trabalho envolve canal bin�rio, sim�trico e sem mem�ria.

% Segundo \cite{Woitaszek:2007}, para sistemas de armazenamento, a codifica��o por apagamento baseada em opera��es simples, tais como XOR RAID e c�digos Tornado, s�o prefer�veis. Apesar de que um mecanismo externo deva ser utilizado para detectar erros, as opera��es de XOR podem ser realizadas rapidamente e resultar em alto \emph{throughput} das opera��es de codifica��o e decodifica��o.

% As codifica��es de blocos que ser�o estudadas e implementadas s�o: c�digos Reed-Solomon (modelo 3 e proposta 4) e c�digos Tornado (proposta 5). S�o algoritmos muito estudados e com conhecidas implementa��es.

\subsection{Contribui��es deste trabalho}

\subsubsection*{\emph{Overview} de Codifica��o por Apagamento}

A revis�o bibliogr�fica desse tema tem exigido tempo e dedica��o, devido a exist�ncia de poucas pesquisas experimentais publicadas sobre o tema. Uma dificuldade encontrada por quem estuda codifica��o por apagamento � que n�o existe uma nomenclatura unificada \cite{Plank:2009}. Tamb�m segundo \cite{CS540:2010}, existem poucos pesquisadores que s�o programadores de sistemas e que fazem propostas neste tema.

% Uma classifica��o para c�digos de blocos pode ser encontrada na figura~\ref{fig4:cbc}.

%    \begin{figure}[h]
%      \centering
%      \includegraphics[scale=1]{blockcodes.png}
%      \caption{Uma classifica��o para c�digos de blocos \cite{MathWorks:2010}}
%      \label{fig4:cbc}
%    \end{figure} 

Uma classifica��o para c�digos de blocos pode ser encontrada em \cite{MathWorks:2010}.

\subsubsection{Submeter as altera��es e sugest�es como contribui��o ao Hadoop}

As altera��es e sugest�es para uso das codifica��es por apagamento no
Hadoop ser�o propostas a comunidade por meio do site da Apache
\emph{Software Foundation}, como foi proposta a vers�o inicial de
codifica��o por apagamento no HDFS~\cite{HDFS-503:2010} e a segunda
vers�o com c�digos Reed-Solomon~\cite{MR-1969:2010}.



%\input{plano}

%\singlespacing
\setstretch{1.5}
\bibliography{celina}

\end{document}
